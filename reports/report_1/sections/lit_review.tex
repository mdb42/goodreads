% Use in social media
Sentiment analysis has been used in machine learning and deep learning contexts for a variety of applications.
Chandra and Jana have found that sentiment analysis can be an effective tool when assessing the general public's
feelings about products and topics on social media \cite{chandra2020}. Going beyond machine learning, it seems
that using deep learning could lead to more accurate analysis, which presents an interesting area to extend
this project into in the future. Additionally, given the social media-like nature of Goodreads, it will be
interesting to see if sentiment analysis is just as effective for Goodreads as it has for Twitter, despite
differences in approach.

% Use in detecting emotions in the plot of a novel
Tangential to Goodreads, sentiment analysis seems to have some interesting use cases when determining structure of
a novel, particularly novels that may not use a traditional plot structure. Elkins and Chun found that sentiment
analysis when paired with manual close reading can result in new critiques of literature \cite{Elkins2019}. Despite
some issues in the \texttt{Syuzhet.R} library used for sentiment analysis, the researchers still found interesting
new insights and emotional arcs in the novel that may be missed by other readers. However, it was noted that the
library used struggled with properly assigning sentimental scores to certain kinds of grammar, like negation,
capitalization, and even emojis. Since we're expecting a much more informal style of writing in Goodreads
reviews, we'll have to monitor how our approach handles situations where excessive punctuation, emojis, and
incorrect spelling/grammar are present.

% More use in social media; pitfall of analyzing more in-depth approaches
Beyond emotional analysis, sentiment analysis could be used to gain insights on those participating in the
conversation(s), too. Sokolova and Bobicev attempted to use sentiment analysis on medical forums to evaluate
the presence of the ``echo chamber effect'' in those forums \cite{Sokolova2020}. However, the authors had some
difficulty finding properly-labeled data to train their model effectively to make these sorts of analyses.
While Goodreads reviews often come with a star rating that can be used as a label, great care needs to be
taken in what kinds of conclusions can be drawn from our approach beyond predicting a star rating.

% Flexibility of sentiment analysis; ASSUMES THAT APPLICATION TO MOVIE REVIEWS IS STILL POSSIBLY HAPPENING
Given that data availability can be an issue in just about any application where a sentiment analyzer may
be used, some researchers have been looking in to making more general-purpose models to improve performance
when used in subjects unrelated to the model's test set. SentiX, a cross-domain sentiment analysis model, was
proposed by Zhou et. al. to be used on several domains of user reviews without the need for fine-tuning the model
along the way \cite{Zhou2020}. This model beats most BERT-based models and several other models (excluding the domain that other
models were trained on) while being trained on less samples than other models. This experiment seems \textit{far} more
complex than ours will be, demonstrating the difficulty in constructing such a model while avoiding overfitting
on the domain that the model is trained on.

% Naive Bayes performance
That being said, Na\"ive Bayes still seems to be a valid framework to build a sentiment analysis model around for
internet user reviews. Quadri and Selvakumar used Na\"ive Bayes to develop a model that analyzes cross-domain
reviews, achieving an accuracy range of 76\% to 99\% against a set of different domains and review websites 
\cite{Quadri2020}. It is noted that the results could be better if the model used other variants of Na\"ive Bayes
for certain domains over others (Multinomial and Bernoulli specifically), but the results achieved here seem to make
a promising case for the effectiveness of our approach in both the trained Goodreads domain and a different
``target'' domain.

% Importance of feature selection
Feature selection is another important component in developing a sentiment analysis model. These are used to make
sure the model is getting relevant information from any given review to give it the best chance to make the correct
prediction on the sentiment of the review. Several sentiment analysis models used in other published journals use
feature selection methods that information retrieval systems tend to lean on, like document frequency, chi-squared ($\mathcal{X}^2$),
odds ratio, and clustering \cite{Hung2015}. While a classifier like Na\"ive Bayes is the decision maker of the model, feature selection
reduces the amount of input to a more reasonable amount, ideally improving performance of the model and resulting
in a more accurate model overall.