Our project primarily builds upon the work of the curators of the UCSD Goodreads dataset. In their first paper, Wan and McAuley \cite{wan2018} introduce chainRec, a recommendation system that models user interactions as sequences of increasingly committed behaviors. This paper is significant to our work not only for the comprehensive Goodreads dataset it provides—comprising over 225 million user-item interactions—but also for its conceptual framework of viewing user engagement as structured sequences rather than independent events. The authors specifically request citation of this work when using their dataset.

In a follow-up study, Wan et al. \cite{wan2019} developed SpoilerNet, a neural model trained on 1.4 million Goodreads reviews to detect spoilers at the sentence level. This research demonstrates that users exhibit stable, quantifiable patterns in their reviewing behavior, offering a methodological precedent for our user-level sentiment analysis. The authors specifically request citation of this work when using their dataset.

For sentiment analysis methodology, Monika and Chooralil \cite{monika2023} offer a comprehensive survey of techniques, providing a systematic framework for the analytical pipeline from data collection through interpretation.

Reagle \cite{reagle2015} examines online comment culture across platforms, developing a taxonomy of commenting behaviors with six functions: informing, improving, manipulating, alienating, shaping identity, and perplexing. While not focused specifically on literary reviewing, Reagle's typology offers a valuable framework for categorizing reviewers and understanding how digital environments may amplify certain reviewing behaviors.

% Use in social media
Sentiment analysis has been used in machine learning and deep learning contexts for a variety of applications.
Chandra and Jana have found that sentiment analysis can be an effective tool when assessing the general public's
feelings about products and topics on social media \cite{chandra2020}. Going beyond machine learning, it seems
that using deep learning could lead to more accurate analysis, which presents an interesting area to extend
this project into in the future. Additionally, given the social media-like nature of Goodreads, it will be
interesting to see if sentiment analysis is just as effective for Goodreads as it has for Twitter, despite
differences in approach.

% Use in detecting emotions in the plot of a novel
Tangential to Goodreads, sentiment analysis seems to have some interesting use cases when determining structure of
a novel, particularly novels that may not use a traditional plot structure. Elkins and Chun found that sentiment
analysis when paired with manual close reading can result in new critiques of literature \cite{Elkins2019}. Despite
some issues in the \texttt{Syuzhet.R} library used for sentiment analysis, the researchers still found interesting
new insights and emotional arcs in the novel that may be missed by other readers. However, it was noted that the
library used struggled with properly assigning sentimental scores to certain kinds of grammar, like negation,
capitalization, and even emojis. Since we're expecting a much more informal style of writing in Goodreads
reviews, we'll have to monitor how our approach handles situations where excessive punctuation, emojis, and
incorrect spelling/grammar are present.

% More use in social media; pitfall of analyzing more in-depth approaches
Beyond emotional analysis, sentiment analysis could be used to gain insights on those participating in the
conversation(s), too. Sokolova and Bobicev attempted to use sentiment analysis on medical forums to evaluate
the presence of the ``echo chamber effect'' in those forums \cite{Sokolova2020}. However, the authors had some
difficulty finding properly-labeled data to train their model effectively to make these sorts of analyses.
While Goodreads reviews often come with a star rating that can be used as a label, great care needs to be
taken in what kinds of conclusions can be drawn from our approach beyond predicting a star rating.

% Flexibility of sentiment analysis; ASSUMES THAT APPLICATION TO MOVIE REVIEWS IS STILL POSSIBLY HAPPENING
Given that data availability can be an issue in just about any application where a sentiment analyzer may
be used, some researchers have been looking in to making more general-purpose models to improve performance
when used in subjects unrelated to the model's test set. SentiX, a cross-domain sentiment analysis model, was
proposed by Zhou et. al. to be used on several domains of user reviews without the need for fine-tuning the model
along the way \cite{Zhou2020}. This model beats most BERT-based models and several other models (excluding the domain that other
models were trained on) while being trained on less samples than other models. This experiment seems \textit{far} more
complex than ours will be, demonstrating the difficulty in constructing such a model while avoiding overfitting
on the domain that the model is trained on.

% Naive Bayes performance
That being said, Na\"ive Bayes still seems to be a valid framework to build a sentiment analysis model around for
internet user reviews. Quadri and Selvakumar used Na\"ive Bayes to develop a model that analyzes cross-domain
reviews, achieving an accuracy range of 76\% to 99\% against a set of different domains and review websites 
\cite{Quadri2020}. It is noted that the results could be better if the model used other variants of Na\"ive Bayes
for certain domains over others (Multinomial and Bernoulli specifically), but the results achieved here seem to make
a promising case for the effectiveness of our approach in both the trained Goodreads domain and a different
``target'' domain.

% Importance of feature selection
Feature selection is another important component in developing a sentiment analysis model. These are used to make
sure the model is getting relevant information from any given review to give it the best chance to make the correct
prediction on the sentiment of the review. Several sentiment analysis models used in other published journals use
feature selection methods that information retrieval systems tend to lean on, like document frequency, chi-squared ($\mathcal{X}^2$),
odds ratio, and clustering \cite{Hung2015}. While a classifier like Na\"ive Bayes is the decision maker of the model, feature selection
reduces the amount of input to a more reasonable amount, ideally improving performance of the model and resulting
in a more accurate model overall.

% Large-scale analysis of Goodreads reviews
Shahsavari et al. developed a pipeline for extracting character and relationship networks from Goodreads book reviews using Greimasian actant theory and latent graphical models \cite{shahsavari2020}. By aggregating thousands of reviews per novel, their system was able to reconstruct a “consensus narrative framework” with high accuracy. While their focus is narrative rather than sentiment, their work demonstrates the viability of mining large-scale Goodreads review data to infer structured latent knowledge from unstructured user text—supporting our motivation to extract aggregated reviewer patterns from the same platform.

% Linguistic and psychological analysis of Goodreads reviews
Hajibayova performed a linguistic and psychological analysis of over 470,000 Goodreads reviews to examine how users express personal reactions through review text \cite{hajibayova2019}. The findings suggest that reviews are not only evaluative but also performative—shaped by an intent to influence cultural consumption, in line with Bourdieu’s theory of symbolic capital. The prevalence of highly positive language was noted as a potential reliability concern. This work helps ground our interpretation of reviewer sentiment and supports the idea that review tone may reflect social motives as well as genuine opinion.
